\documentclass[xcolor=pdftex,dvipsnames,usenames,table]{beamer}

\usepackage[spanish]{babel}
\usepackage[utf8]{inputenc}
%\usepackage{times}
\usepackage{subfigure}
\usepackage{multirow}
\usepackage{array}
\usepackage{amsmath}

\mode<presentation>
{
  \usetheme{Singapore}
  \usecolortheme{default}
  \setbeamercovered{transparent}

  %\setbeamertemplate{part page}
  %{
    %\begin{centering}
      %\begin{beamercolorbox}[sep=8pt,center]{part title}
        %\usebeamerfont{part title}\insertpart\par
      %\end{beamercolorbox}
    %\end{centering}
  %}
}

\title[]
{Primitivas de Memoria Virtual para Porgramas de Usuario}

\subtitle
{Virtual Memory Primitives for User Programs}

\author[]
{
	Leandro Liptak\\
	Patricio Reboratti\\
	Damián Silvani
}

\institute[UBA]
{
  Programación de Sistemas Operativos\\
  Universidad de Buenos Aires
}
\date
{9 de junio, 2011}

% If you wish to uncover everything in a step-wise fashion, uncomment
% the following command: 
\beamerdefaultoverlayspecification{<+->}

\begin{document}

\begin{frame}
  \titlepage
\end{frame}

\section{Introducción}

\begin{frame}{Memoria Virtual}
  \begin{itemize}
    \item Principalmente, permite
      \begin{itemize}
        \item Extender el espacio de direccionamiento
        \item Compartir páginas entre procesos
        \item Proteger páginas de código como \textit{sólo lectura}
        \item Implementar \texttt{copy-on-write}
      \end{itemize}

    \item Puede tener otras aplicaciones\ldots
    \begin{itemize}
      \item Unix traduce un PF en un SIGSEGV, el usuario puede atraparlo
      \item Se puede hacer algo mejor?
    \end{itemize}
 \end{itemize}
\end{frame}

\begin{frame}{Primitivas de Memoria Virtual}
  
  \begin{itemize}
    \item \texttt{trap}
    \item \texttt{prot1} y \texttt{protN}
    \item \texttt{unprot}
    \item \texttt{dirty}
    \item \texttt{map2}
  \end{itemize}
\end{frame}

\section{Aplicaciones}

\begin{frame}{Memoria virtual compartida}
\end{frame}

\begin{frame}{Checkpointing concurrente}
\end{frame}

\begin{frame}{GC concurrente}
\end{frame}

\begin{frame}{GC generacional}
\end{frame}

\begin{frame}{Heap persistente}
\end{frame}

\begin{frame}{Extensión de direccionamiento}
\end{frame}

\begin{frame}{Compresión de páginas}
\end{frame}

\begin{frame}{Detección de heap overflow}
\end{frame}

\section{Consideraciones}

\begin{frame}{Consistencia de la TLB}
\end{frame}

\begin{frame}{Tamaño de página óptimo}
\end{frame}

\begin{frame}{Acceso a páginas protegidas}
\end{frame}

\begin{frame}{Trap handlers sincrónicos}
\end{frame}

\section{Conclusiones}

\begin{frame}{Conclusión}
\end{frame}

\end{document}
