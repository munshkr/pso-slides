\documentclass[xcolor=pdftex,dvipsnames,usenames,table]{beamer}

\usepackage[spanish]{babel}
\usepackage[utf8]{inputenc}
%\usepackage{times}
\usepackage{subfigure}
\usepackage{multirow}
\usepackage{array}
\usepackage{amsmath}

\mode<presentation>
{
  \usetheme{Singapore}
  \usecolortheme{default}
  \setbeamercovered{transparent}

  %\setbeamertemplate{part page}
  %{
    %\begin{centering}
      %\begin{beamercolorbox}[sep=8pt,center]{part title}
        %\usebeamerfont{part title}\insertpart\par
      %\end{beamercolorbox}
    %\end{centering}
  %}
}

\title[]
{Primitivas de Memoria Virtual para Porgramas de Usuario}

\subtitle
{Virtual Memory Primitives for User Programs}

\author[]
{
	Leandro Liptak\\
	Patricio Reboratti\\
	Damián Silvani
}

\institute[UBA]
{
  Programación de Sistemas Operativos\\
  Universidad de Buenos Aires
}
\date
{9 de junio, 2011}

% If you wish to uncover everything in a step-wise fashion, uncomment
% the following command:
\beamerdefaultoverlayspecification{<+->}

\begin{document}

\begin{frame}
  \titlepage
\end{frame}

\section{Introducción}

\begin{frame}{Memoria Virtual}
  \begin{itemize}
    \item Principalmente, permite
      \begin{itemize}
        \item Extender el espacio de direccionamiento
        \item Compartir páginas entre procesos
        \item Proteger páginas de código como \textit{sólo lectura}
        \item Implementar \texttt{copy-on-write}
      \end{itemize}

    \item Puede tener otras aplicaciones\ldots
    \begin{itemize}
      \item Unix traduce un PF en un SIGSEGV, el usuario puede atraparlo
      \item Se puede hacer algo mejor?
    \end{itemize}
 \end{itemize}
\end{frame}

\begin{frame}{Primitivas de Memoria Virtual}
  \begin{itemize}
    \item \texttt{trap}
    \item \texttt{prot1} y \texttt{protN}
    \item \texttt{unprot}
    \item \texttt{dirty}
    \item \texttt{map2}
  \end{itemize}
\end{frame}

\section{Aplicaciones}

\begin{frame}{Memoria virtual compartida}
  \begin{itemize}
    \item Memoria compartida, distribuida en varias computadoras
    \item Acceso a una memoria compartida coherente
    \item Se divide el espacio en páginas:
    \begin{itemize}
      \item En escritura, si la página reside en otras memorias físicas, se
            busca una copia actualizada y se invalida las otras copias.
      \item Las de sólo lectura pueden estar en varias memorias físicas al mismo
            tiempo.
    \end{itemize}
    \item Funciona similar a la virtualización con swapping:
    \begin{itemize}
      \item Un Page Fault ocurre cuando la página no se encuentra en la propia
            memoria física.
      \item El MMU puede buscar la página en el disco o en otra memoria física.
    \end{itemize}
    \item Necesita \texttt{trap}, \texttt{prot1}, \texttt{unprot} y \texttt{map2}
  \end{itemize}
\end{frame}

\begin{frame}{Checkpointing concurrente}
\end{frame}

\begin{frame}{GC concurrente}
\pause En un Garbage Collector incremental basado en copia como el Baker's GC, la protección de páginas y el manejo de fallos por parte del recolector brinda un mecanismo eficiente de sincronización entre los treads mutadores y el recolector.

\pause Para esto (suponiento un solo thread mutador):
	\pause
	\begin{itemize}
		\item Se crean dos mapeos de las páginas de \emph{from-space} con \textbf{\texttt{map2}}
		\item Para cada página, se protege el mapeo que corresponde al thread del mutador con \textbf{\texttt{protN}}
	\end{itemize}

\pause Ante una referencia a un objeto en \emph{from-space} por parte del mutador, el thread recolector:
	\pause
	\begin{itemize}
		\item Atrapa el fallo con \textbf{\texttt{trap}}
		\item Copia los objetos "vivos" de dicha página a \emph{to-space}
		\item Actualiza los punteros a dichos objetos en el thread mutador
		\item Desprotege la páginas en cuestión con \textbf{\texttt{unprot}}
	\end{itemize}

\end{frame}

\begin{frame}{GC generacional}
\end{frame}

\begin{frame}{Heap persistente}
\end{frame}

\begin{frame}{Extensión de direccionamiento}
\end{frame}

\begin{frame}{Compresión de páginas}
\end{frame}

\begin{frame}{Detección de heap overflow}
\end{frame}

\section{Consideraciones}

\begin{frame}{Consistencia de la TLB}
  \begin{itemize}
    \item Cuando una página se hace más accesible, una TLB desactualizada a lo
          sumo causa un \textit{cache miss}.
    \item Pero al proteger una página, es necesario invalidar la TLB.
    \item En multiprocesadores, el problema es más notorio:
    \begin{itemize}
      \item Se debe interrumpir cada procesador e invalidar sus TLBs.
      \item Costoso por software
    \end{itemize}
    \item Solución: hacer \textit{flush} por lote
    \item La mayoria de los algoritmos usan \texttt{protN}, no \texttt{prot1}
  \end{itemize}
\end{frame}

\begin{frame}{Tamaño de página óptimo}
\end{frame}

\begin{frame}{Acceso a páginas protegidas}
\end{frame}

\begin{frame}{Trap handlers sincrónicos}
\end{frame}

\section{Conclusiones}

\begin{frame}{Conclusión}
\end{frame}

\end{document}
